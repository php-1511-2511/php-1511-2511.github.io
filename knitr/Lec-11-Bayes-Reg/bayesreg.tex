\documentclass[handout]{beamer}\usepackage[]{graphicx}\usepackage[]{color}
%% maxwidth is the original width if it is less than linewidth
%% otherwise use linewidth (to make sure the graphics do not exceed the margin)
\makeatletter
\def\maxwidth{ %
  \ifdim\Gin@nat@width>\linewidth
    \linewidth
  \else
    \Gin@nat@width
  \fi
}
\makeatother

\definecolor{fgcolor}{rgb}{0.345, 0.345, 0.345}
\newcommand{\hlnum}[1]{\textcolor[rgb]{0.686,0.059,0.569}{#1}}%
\newcommand{\hlstr}[1]{\textcolor[rgb]{0.192,0.494,0.8}{#1}}%
\newcommand{\hlcom}[1]{\textcolor[rgb]{0.678,0.584,0.686}{\textit{#1}}}%
\newcommand{\hlopt}[1]{\textcolor[rgb]{0,0,0}{#1}}%
\newcommand{\hlstd}[1]{\textcolor[rgb]{0.345,0.345,0.345}{#1}}%
\newcommand{\hlkwa}[1]{\textcolor[rgb]{0.161,0.373,0.58}{\textbf{#1}}}%
\newcommand{\hlkwb}[1]{\textcolor[rgb]{0.69,0.353,0.396}{#1}}%
\newcommand{\hlkwc}[1]{\textcolor[rgb]{0.333,0.667,0.333}{#1}}%
\newcommand{\hlkwd}[1]{\textcolor[rgb]{0.737,0.353,0.396}{\textbf{#1}}}%
\let\hlipl\hlkwb

\usepackage{framed}
\makeatletter
\newenvironment{kframe}{%
 \def\at@end@of@kframe{}%
 \ifinner\ifhmode%
  \def\at@end@of@kframe{\end{minipage}}%
  \begin{minipage}{\columnwidth}%
 \fi\fi%
 \def\FrameCommand##1{\hskip\@totalleftmargin \hskip-\fboxsep
 \colorbox{shadecolor}{##1}\hskip-\fboxsep
     % There is no \\@totalrightmargin, so:
     \hskip-\linewidth \hskip-\@totalleftmargin \hskip\columnwidth}%
 \MakeFramed {\advance\hsize-\width
   \@totalleftmargin\z@ \linewidth\hsize
   \@setminipage}}%
 {\par\unskip\endMakeFramed%
 \at@end@of@kframe}
\makeatother

\definecolor{shadecolor}{rgb}{.97, .97, .97}
\definecolor{messagecolor}{rgb}{0, 0, 0}
\definecolor{warningcolor}{rgb}{1, 0, 1}
\definecolor{errorcolor}{rgb}{1, 0, 0}
\newenvironment{knitrout}{}{} % an empty environment to be redefined in TeX

\usepackage{alltt}
%\documentclass[]{beamer}
\usepackage{graphicx}
\usepackage{amsmath,amssymb,array,comment,eucal}
\usepackage{amsmath,amssymb,array,eucal}
\usepackage{xcolor}
\definecolor{beamer@blendedblue}{RGB}{86,155,189}
\definecolor{myblue}{RGB}{12,76,138}
\setbeamercolor{structure}{fg=myblue}
\definecolor{Ftitle}{RGB}{12,76,138}
\definecolor{Descitem}{RGB}{238,238,244}
\definecolor{StdTitle}{RGB}{12,76,138}
\definecolor{StdBody}{RGB}{213,24,0}
\definecolor{StdBody}{RGB}{213,24,0}

\definecolor{AlTitle}{RGB}{255, 190, 190}
\definecolor{AlBody}{RGB}{213,24,0}

\definecolor{ExTitle}{RGB}{201, 217, 217}
\definecolor{ExBody}{RGB}{213,24,0}

\setbeamercolor{frametitle}{fg = Ftitle}
\setbeamercolor{title}{fg = Ftitle}
\setbeamercolor{item}{fg = Ftitle}
\setbeamercolor{subitem}{fg = Ftitle}
\setbeamercolor{subsubitem}{fg = Ftitle}
\setbeamercolor{description item}{fg = myblue}
\setbeamercolor{titlelike}{fg=myblue}
\newcommand{\e}{\mathbf{e}}
\renewcommand{\P}{\textsf{P}}
\newcommand{\R}{\textsf{R}}
\newcommand{\mat}[1] {\mathbf{#1}}
%\newcommand{\ind}{\mathrel{\mathop{\sim}\limits^{\mathit{ind}}}}
%\newcommand{\iid}{\mathrel{\mathop{\sim}\limits^{\mathit{iid}}}}
\newcommand{\E}{\textsf{E}}
\newcommand{\SE}{\textsf{SE}}
\newcommand{\SSE}{\textsf{SSE}}
\renewcommand{\SS}{\textsf{SS}}
\newcommand{\MSE}{\textsf{MSE}}
\newcommand{\SSR}{\textsf{SSR}}
\newcommand{\Be}{\textsf{Beta}}
\newcommand{\St}{\textsf{St}}
%\newcommand{\C}{\textsf{C}}
\newcommand{\GDP}{\textsf{GDP}}
\newcommand{\NcSt}{\textsf{NcSt}}
\newcommand{\Bin}{\textsf{Bin}}
\newcommand{\NB}{\textsf{NegBin}}
\renewcommand{\NG}{\textsf{NG}}
\newcommand{\N}{\textsf{N}}
\newcommand{\Ber}{\textsf{Ber}}
\newcommand{\Poi}{\text{Poi}}
\newcommand{\Gam}{\textsf{Gamma}}
\newcommand{\Gm}{\textsf{G}}
\newcommand{\Un}{\textsf{Unif}}
\newcommand{\Ex}{\textsf{Exp}}
\newcommand{\DE}{\textsf{DE}}
\newcommand{\tr}{\textsf{tr}}
\newcommand{\cF}{{\cal{F}}}
\newcommand{\cL}{{\cal{L}}}
\newcommand{\cI}{{\cal{I}}}
\newcommand{\cB}{{\cal{B}}}
\newcommand{\cP}{{\cal{P}}}
\newcommand{\bbR}{\mathbb{R}}
\newcommand{\bbN}{\mathbb{N}}
\newcommand{\pperp}{\mathrel{{\rlap{$\,\perp$}\perp\,\,}}}
\newcommand{\OFP}{(\Omega,\cF, \P)}
\newcommand{\eps}{\boldsymbol{\epsilon}}
\newcommand{\1}{\mathbf{1}_n}
\newcommand{\gap}{\vspace{8mm}}
\newcommand{\ind}{\mathrel{\mathop{\sim}\limits^{\rm ind}}}
\newcommand{\simiid}{\ensuremath{\mathrel{\mathop{\sim}\limits^{\rm
iid}}}}
\newcommand{\eqindis}{\ensuremath{\mathrel{\mathop{=}\limits^{\rm D}}}}
\newcommand{\iid}{\textit{i.i.d.}}
\newcommand{\SSZ}{S_{zz}}
\newcommand{\SZW}{S_{zw}}
\newcommand{\Bias}{\textsf{Bias}}
\newcommand{\Var}{\textsf{Var}}
\newcommand{\corr}{\textsf{corr}}
\newcommand{\diag}{\textsf{diag}}
\newcommand{\var}{\textsf{var}}
\newcommand{\Cov}{\textsf{Cov}}
\newcommand{\Sam}{{\cal S}}
\def\H{\mathbf{H}}
\newcommand{\I}{\mathbf{I}}
\newcommand{\Y}{\mathbf{Y}}
\newcommand{\tY}{\tilde{\mathbf{Y}}}
\newcommand{\Yhat}{\hat{\mathbf{Y}}}
\newcommand{\Yobs}{\mathbf{Y}_{{\cal S}}}
\newcommand{\barYobs}{\bar{Y}_{{\cal S}}}
\newcommand{\barYmiss}{\bar{Y}_{{\cal S}^c}}
\def\bv{\mathbf{b}}
\def\X{\mathbf{X}}
\def\tX{\tilde{\mathbf{X}}}
\def\x{\mathbf{x}}
\def\xbar{\bar{\x}}
\def\Xg{\mathbf{X}_{\boldsymbol{\gamma}}}
\def\Ybar{\bar{Y}}
\def\ybar{\bar{y}}
\def\y{\mathbf{y}}
\def\Yf{\mathbf{Y_f}}
\def\W{\mathbf{W}}
\def\w{\mathbf{w}}
\def\U{\mathbf{U}}
\def\V{\mathbf{V}}
\def\Q{\mathbf{Q}}
\def\Z{\mathbf{Z}}
\def\z{\mathbf{z}}
\def\v{\mathbf{v}}
\def\u{\mathbf{u}}

\def\zero{\mathbf{0}}
\def\one{\mathbf{1}}
\newcommand{\taub}{\boldsymbol{\tau}}
\newcommand{\betav}{\boldsymbol{\beta}}
\newcommand{\alphav}{\boldsymbol{\alpha}}
\newcommand{\A}{\mathbf{A}}
\def\a{\mathbf{a}}
\newcommand{\B}{\mathbf{B}}
\def\b{\boldsymbol{\beta}}
\def\bhat{\hat{\boldsymbol{\beta}}}
\def\tb{\tilde{\boldsymbol{\beta}}}
\def\bg{\boldsymbol{\beta_\gamma}}
\def\bgnot{\boldsymbol{\beta_{(-\gamma)}}}
\def\mub{\boldsymbol{\mu}}
\def\tmub{\tilde{\boldsymbol{\mu}}}
\def\muhat{\hat{\boldsymbol{\mu}}}
\def\t{\boldsymbol{\theta}}
\def\tk{\boldsymbol{\theta}_k}
\def\tj{\boldsymbol{\theta}_j}
\def\Mk{\boldsymbol{{\cal M}}_k}
\def\M{{{\cal M}}}
\def\Mj{{{\cal M}}_j}
\def\Mi{{{\cal M}}_i}
\def\Mg{{\cal M}_\gamma}
\def\Mnull{{\cal M}_{N}}
\def\gMPM{\boldsymbol{\gamma}_{\text{MPM}}}
\def\gHPM{\boldsymbol{\gamma}_{\text{HPM}}}
\def\Mfull{\boldsymbol{{\cal M}}_{F}}
\def\tg{\boldsymbol{\theta}_{\boldsymbol{\gamma}}}
\def\g{\boldsymbol{\gamma}}
\def\eg{\boldsymbol{\eta}_{\boldsymbol{\gamma}}}
\def\G{\mathbf{G}}
\def\cM{\cal M}
\def\D{\Delta}
\def \shat{{\hat{\sigma}}^2}
\def\uv{\mathbf{u}}
\def\l {\lambda}
\def\d{\delta}
\def\Sigmab{\boldsymbol{\Sigma}}
\def\Lambdab{\boldsymbol{\Lambda}}
\def\lambdab{\boldsymbol{\lambda}}
\def\Mg{{\cal M}_\gamma}
\def\S{{\cal{S}}}
\def\qg{p_{\boldsymbol{\gamma}}}
\def\pg{p_{\boldsymbol{\gamma}}}
\def\t{\boldsymbol{\theta}}
\def\T{\boldsymbol{\Theta}}

\usepackage{verbatim}

%\usetheme{Warsaw}
%\usecolortheme{orchid}
\title{Bayesian Estimation in Linear Models}
\institute{Merlise Clyde}
\author{STA521 Linear Models Duke University}
\date{\today}
%\logo{duke.eps}
\IfFileExists{upquote.sty}{\usepackage{upquote}}{}
\begin{document}
\maketitle

\begin{frame}
  \frametitle{Bayesian Estimation}
  Model $\Y = \X \b + \eps$  with $\eps \sim \N(\zero_n , \sigma^2
  \I_n)$ \pause
is equivalent to
$$
\Y \sim \N(\X \b, \I_n/\phi)
$$
\pause
 $\phi = 1/\sigma^2$ is the {\it precision}.
\pause

\vspace{14pt}
In the  Bayesian paradigm describe uncertainty about unknown
parameters using probability distributions
\pause
\begin{itemize}
\item  Prior Distribution $p(\b, \phi)$ describes uncertainty about
  parameters prior to seeing the data \pause
\item Posterior Distribution $p(\b, \phi \mid \Y)$ describes
  uncertainty about the parameters after updating  believes given the
  observed data \pause
\item updating rule is based on Bayes Theorem
$$p(\b, \phi \mid \Y) \propto \cL(\b, \phi) p(\b, \phi)$$
\pause
\end{itemize}
 reweight prior beliefs by likelihood of parameters  under observed data
\end{frame}

\begin{frame}
  \frametitle{Posterior}
  Posterior is obtained by conditional distribution theory \pause

  Let $\t = (\b, \phi)^T$ \pause
\begin{eqnarray*}
  p(\t \mid \Y ) & = & \frac{p(\Y \mid \t) p(\t) }
{ \int_{\T} p(\Y \mid \t)   p(\t) \, d\t} \pause \\
& = &  \frac{p(\Y, \t)}{p(\Y)}
\end{eqnarray*}
\pause
$p(\Y)$, the normalizing constant, is the marginal distribution of the data.
\pause
\vfill

Easiest to work with Bayes Theorem in proportional form and then
identify the normalizing constant.
\end{frame}

\begin{frame}
  \frametitle{Prior Distributions}
Factor joint prior distribution  $$p(\b, \phi) = p(\b \mid \phi) p(\phi)$$
\pause

Convenient choice is to take \pause
\begin{itemize}
\item $\b \mid \phi \sim \N(b_0, \Phi_0^{-1}/\phi)$ where $b_0$ is the prior
  mean and $\Phi^{-1}/\phi$ is the prior covariance of $\b$ \pause
 $$ p(\b \mid \phi) = (2 \pi)^{-p/2} \phi^{p/2} | \Phi_0|^{1/2}
  \exp\left(- \frac{\phi}{2} (\b - b_0)^T \Phi_0 (\b - b_0)\right)$$
\item $\phi \sim \G( \nu_0/2, \SS_0/2)$  with $\E(\sigma^2) =
  \SS_0/(\nu_0 - 2)$ \pause
$$p(\phi) = \frac{1}{\Gamma{(\nu_0/2)}}
\left(\frac{\SS_0}{2} \right)^{\nu_0/2}
\phi^{\nu_0/2 - 1}
 e^{- \phi \SS_0/2}
 $$\pause
\item $(\b, \phi)^T \sim \NG(\bv_0, \Phi_0, \nu_0, \SS_0)$ \pause
\item Conjugate  ``Normal-Gamma'' family implies \pause
$$(\b, \phi)^T \mid \Y \sim \NG(\bv_n, \Phi_n, \nu_n, \SS_n)$$
\end{itemize}
\end{frame}
\begin{frame}
  \frametitle{Posterior Distribution}
  \begin{eqnarray*}
     p(\b, \phi \mid \Y) &\propto&  \cL(\b, \phi) p(\b \mid \phi) p(\phi)
    \pause \\
 & = & c_Y\, \phi^{n/2} e^{- \frac \phi 2 (\Y - \X \b)^T (\Y - \X \b)} \pause \\
& & c_{\b}\, |\phi \Phi_0|^{1/2} e^{- \frac{\phi}{2} (\b - \bv_0)^T \Phi_0 (\b - \bv_0) }  \pause \\
& &  c_\phi \, \phi^{\frac{nu_0}{2} - 1} e^{-\phi \SS_0/2}
  \end{eqnarray*}
Write $\Phi_0 = \X_0^T \X_0$ \pause \\(by Spectral Theorem $\Phi = \U \Lambdab
\U^T$ take $\X_0 = \Lambdab^{1/2} \U^T$)
\pause
\begin{eqnarray*}
p(\b \mid \phi) & \propto & \phi^p/2 e^{- \frac{\phi}{2} (\b - \bv_0)^T
  \X_0^T\X_0 (\b - \bv_0) } \pause \\
 & = & \phi^{p/2} e^{- \frac{\phi}{2} (\X_0\b - \X_0\bv_0)^T
  (\X_0\b - \X_0\bv_0) } \pause
\end{eqnarray*}
Looks like the likelihood from a prior sample of size $p$ with prior
design $\X_0$ and $\Y_0 = \X_o \bv_0$

\end{frame}
\begin{frame}
  \frametitle{Prior Data}
  Let
 $$\tY = \left[
   \begin{array}{l}
     \Y \\
\Y_0
   \end{array}
\right] \in \bbR^{n+p}  \qquad \tX = \left[
   \begin{array}{l}
     \Y \\
\X_0
   \end{array}
\right]  (n+p \times p)  $$  \pause
then
$$p(\b, \phi \mid \Y) \propto p(\tY \mid \b, \phi) p(\phi)$$
where  $\tY \sim \N( \tX \b, \I_{n+p}/\phi)$
\pause

\vspace{24pt}
Likelihood proportional to sampling model of ``augmented'' data
\pause
\begin{eqnarray*}
 p(\b, \phi \mid \Y) &\propto&  \phi^{\frac n 2} e^{- \frac \phi 2 (\Y - \X \b)^T (\Y - \X \b)}
 |\phi \Phi_0|^{\frac 1 2} e^{- \frac{\phi}{2} (\X_0\b - \X_0\bv_0)^T  (\X_0\b - \X_0\bv_0) }  \\
& &  \phi^{\frac{nu_0}{2} - 1} e^{-\phi \SS_0/2} \pause \\
 &\propto&  \phi^{\frac {n + p}{ 2}} e^{- \frac \phi 2 (\tY - \tX \b)^T (\tY - \tX \b)}
\phi^{\frac{nu_0}{2} - 1} e^{-\phi \frac{ \SS_0}{ 2}}
  \end{eqnarray*}
\end{frame}
\begin{frame}
  \frametitle{Decompose}
  Let $\P_{\tX} = \tX (\tX^T\tX)^{-1} \tX^T$ \pause -  this is an orthogonal
  projection on $C(\tX)$ \pause
 $$\tY^T\tY = \tY^T\P_{\tX}\tY +
  \tY^T(\I_{n+p} - \P_{\tX} )\tY$$

\pause

\begin{eqnarray*}
 p(\b, \phi \mid \Y)
 &\propto&  \phi^{\frac {n + p}{ 2}} e^{- \frac \phi 2 (\tY - \tX \b)^T (\tY - \tX \b)}
\phi^{\frac{nu_0}{2} - 1} e^{-\phi \frac{ \SS_0}{ 2}} \pause \\
 & = &  \phi^{\frac {n + p}{ 2}} e^{- \frac \phi 2 (\tY - \tX \b)^T
   \P_{\tX} (\tY - \tX \b)} \pause \\
 & & e^{- \frac \phi 2 (\tY - \tX \b)^T (\I_{n+p} - \P_{\tX}) (\tY -
   \tX \b)} \pause \\
& & \phi^{\frac{nu_0}{2} - 1} e^{-\phi \frac{ \SS_0}{ 2}} \pause \\
& = &  \phi^{\frac {p}{ 2}} e^{- \frac{\phi} {2}
(\P_{\tX}\tY - \tX \b)^T (\P_{\tX}\tY - \tX \b)} \pause \\
& & \phi^{\frac{n + nu_0}{2} - 1} e^{- \frac \phi 2 \tY^T (\I_{n +p}
  -\P_{\tX}) \tY}  e^{-\phi \frac{ \SS_0}{ 2}}
  \end{eqnarray*}

\end{frame}
\begin{frame}
  \frametitle{Continued}
$\P_{\tX} \tY = \tX \tb$  Maximum a Posteriori (MAP) estimator
\begin{eqnarray*}
 p(\b, \phi \mid \Y)
 &\propto&   \phi^{\frac {p}{ 2}} e^{- \frac{\phi} {2}
(\tX\tb - \tX \b)^T (\tX\tb - \tX \b)}\\
& & \phi^{\frac{n + \nu_0}{2} - 1} e^{-  \phi \frac{ \tY^T (\I_{n +p}
  -\P_{\tX}) \tY +  \SS_0}{ 2}} \pause \\
&\propto&   \phi^{\frac {p}{ 2}} e^{- \frac{\phi} {2}
(\b - \alert{\tb})^T \alert{\tX^T\tX}(\b - \alert{\tb})} \,
 \phi^{\alert{\frac{n + \nu_0}{2}} - 1} e^{-  \phi \alert{ \frac{ \tY^T (\I_{n +p}
  -\P_{\tX}) \tY +  \SS_0}{ 2}}}
  \end{eqnarray*}
Read off distributions!

\begin{eqnarray*}
  \b \mid \phi, \Y & \sim & \N( \tb, (\phi \tX^T \tX)^{-1}) \pause \\
  \phi \mid \Y & \sim & \G\left(\frac{n + \nu_0}{2},  \frac{\tY^T (\I_{n +p}
  -\P_{\tX}) \tY +  \SS_0}{ 2} \right)
\end{eqnarray*}
\end{frame}
\begin{frame}
  \frametitle{Simplify Distribution of $\b$}

Posterior Precision $\phi \tX^T\tX$ \pause
  \begin{eqnarray*}
\tX^T\tX &= &\X^T\X + \X_0^T\X_0 \pause \\
         & = &\X^T\X + \Phi_0 \pause \\
\Phi_n & = & \X^T\X + \Phi_0 \pause
\end{eqnarray*}
sum of precision from data plus prior precision \pause

\begin{eqnarray*}
  \tb & = &  (\tX^T\tX)^{-1} \tX^T\tY \pause \\
 & = &  (\X^T\X + \Phi_0)^{-1} (\X^T\Y + \X_0^T \X_0 \bv_0) \pause \\
 & = &  (\X^T\X + \Phi_0)^{-1}[ (\X^T\X) (\X^T\X)^{-1} \X^T\Y + \Phi_0
 \bv_0] \pause \\
& = & (\X^T\X + \Phi_0)^{-1}[ (\X^T\X) \bhat + \Phi_0
 \bv_0] \pause \\
\bv_n & = & (\X^T\X + \Phi_0)^{-1}[ (\X^T\X) \bhat + \Phi_0
 \bv_0]  \pause
\end{eqnarray*}

Conditional Posterior for $\b \mid \phi, \Y  \sim  \N( \bv_n, \phi^{-1} \Phi_n^{-1})$

\end{frame}
\begin{frame}
  \frametitle{Simplify Posterior Distribution for $\phi$}
  \begin{eqnarray*}
  \phi \mid \Y & \sim & \G\left(\frac{n + \nu_0}{2},  \frac{\tY^T (\I_{n +p}
  -\P_{\tX}) \tY +  \SS_0}{ 2} \right) \pause \\
& & \\
\tY^T (\I_{n +p}   -\P_{\tX}) \tY & = & \| \tY - \tX \bv_n \|^2 \pause \\
& = &  \| \Y - \X \bv_n \|^2 +  \| \Y_0 - \X_0 \bv_n \|^2 \pause \\
& = &  \| \Y - \X \bv_n \|^2 +  (\X_0\bv_0 - \X_0 \bv_n)^T  (\X_0\bv_0
- \X_0 \bv_n) \pause \\
& = &  \| \Y - \X \bv_n \|^2 +  (\bv_0 - \bv_n)^T(\X_o^T\X_0)(\bv_0
- \bv_n) \pause \\
& = &  \| \Y - \X \bv_n \|^2 +  (\bv_0 - \bv_n)^T\Phi_o(\bv_0
- \bv_n) \pause\\
\nu_n & = & n + \nu_0\pause \\
\end{eqnarray*}
Posterior for $\phi$
$$ \phi \mid \Y \sim \G\left(\frac{\nu_n}{2},  \frac{\| \Y - \X \bv_n \|^2 +  (\bv_0 - \bv_n)^T\Phi_o(\bv_0
- \bv_n) +  \SS_0}{ 2} \right) $$
\end{frame}

\begin{frame}
  \frametitle{Marginal Distribution from Normal--Gamma }
  \begin{theorem}
    Let  $\t \mid \phi \sim \N(m, \frac{1}{\phi} \Sigma)$ and $\phi \sim
    \G(\nu/2, \nu \shat/2)$. Then  $\t$ ($p \times 1)$ has a $p$
    dimensional multivariate $t$ distribution $$\t \sim t_\nu( m,
    \shat \Sigma )$$ with density
$$p(\t) \propto  \left[ 1 + \frac{1}{\nu}  \frac{ (\t - m)^T
    \Sigma^{-1} (\t - m)}{\shat} \right]^{- \frac{p + \nu}{2}}$$
  \end{theorem}
\end{frame}
\begin{frame}
  \frametitle{Derivation}
Marginal density  $p(\t) = \int p(\t \mid \phi) p(\phi) \, d\phi$
\pause
\begin{eqnarray*}
  p(\t) & \propto & \int |\Sigma/\phi|^{-1/2}
e^{- \frac{\phi}{2} (\t - m)^T
    \Sigma^{-1} (\t - m)}  \phi^{\nu/2 - 1} e^{- \phi \frac{\nu
      \shat}{2}}\, d \phi \pause \\
  & \propto & \int \phi^{p/2} \phi^{\nu/2 - 1}
e^{- \phi \frac{(\t - m)^T
    \Sigma^{-1} (\t - m)+  \nu
      \shat}{2}}\, d \phi \pause \\
 & \propto & \int \phi^{\frac{p +\nu}{2} - 1}
e^{- \phi \frac{(\t - m)^T
    \Sigma^{-1} (\t - m)+  \nu
      \shat}{2}} \, d \phi \pause \\
& = & \Gamma((p + \nu)/2 ) \left( \frac{(\t - m)^T
    \Sigma^{-1} (\t - m)+  \nu
      \shat}{2} \right)^{- \frac{p + \nu}{2}} \pause \\
& \propto &  \left( (\t - m)^T
    \Sigma^{-1} (\t - m)+  \nu
      \shat \right)^{- \frac{p + \nu}{2}} \pause \\
& \propto &  \left( 1 + \frac{1}{\nu}  \frac{(\t - m)^T
    \Sigma^{-1} (\t - m)}{\shat}
       \right)^{- \frac{p + \nu}{2}}
\end{eqnarray*}
\end{frame}
\begin{frame}
  \frametitle{Marginal Posterior Distribution of $\b$}
  \begin{eqnarray*}
\b \mid \phi, \Y  & \sim & \N( \bv_n, \phi^{-1} \Phi_n^{-1}) \pause \\
 \phi \mid \Y & \sim & \G\left(\frac{\nu_n}{2},  \frac{\SS_n}{ 2} \right) \\
    & \sim & \G\left(\frac{\nu_n}{2},  \frac{\nu_n \shat}{ 2} \right)
    \text{ where } \shat = \SS_n/\nu_n  \text{ (Bayesian MSE)}
  \end{eqnarray*}
\pause


Then the marginal posterior distribution of $\b$ is
$$
\b  \mid \Y \sim t_{\nu_n} (\bv_n, \shat \Phi_n^{-1})
$$ \pause


Any linear combination $\lambda^T\b$
$$\lambda^T\b  \mid \Y \sim t_{\nu_n}
(\lambda^T\bv_n, \shat \lambda^T\Phi_n^{-1}\lambda)$$ has a univariate
$t$ distribution with $\v_n$ degrees of freedom

\end{frame}


\begin{frame} \frametitle{Posterior Distribution of $\mu_i = \x^T_i \b$}

\begin{itemize}
\item Use a limiting prior distribution: $p(\b, \phi) \propto \phi^{-1}$
$$b_0 = 0, \SS_0 = 0, \Phi_0 = 0, \nu_0 =  -(p+1)$$

\item joint posterior distribution depends on MLE's
\begin{align*}
\b \mid \phi, \Y & \sim  \N\left(\hat{\b}, \phi^{-1}(\X^T\X)^{-1}\right) \\
\phi \mid \Y & \sim  \G((n - p -1)/2, (n - p - 1)\hat \sigma^2/2) \\
\mu_i \mid \Y & \sim t_{n -p -1}(\x^T_i\hat{\b},  \hat{\sigma}^2 \x_i^T (\X^T\X)^{-1}\x_i)
\end{align*}
\item Same distribution as frequentists:
$$P( \mu  \in  (
\x_i^T\hat{\b} \pm t_{\alpha/2} \hat{\sigma}^2 h_{ii})
\mid \Y) = 1 - \alpha $$
\end{itemize}

\end{frame}


\begin{frame}
  \frametitle{Predictive Distribution}
Suppose $\Y^* \mid \b, \phi \sim \N(\X^* \b , \I_f/\phi)$  and is conditionally
independent of $\Y$ given $\b$ and $\phi$ \pause
\vspace{18pt}

What is the predictive distribution of $\Y^* \mid \Y$? \pause

\vspace{18pt}
$\Y^* = \X^* \b + \eps^*$ and $\eps^*$ is independent of $\Y$ given
$\phi$ \pause

\begin{eqnarray*}
\X^* \b + \eps^* \mid \Y, \phi & \sim & \N(\X^*\bv_n, (\X^{*} \Phi_n^{-1} \X^{*T}
+ \I)/\phi)  \pause \\
\Y^* \mid \phi, \Y & \sim & \N(\X^*\bv_n, (\X^{*T} \Phi_n^{-1} \X^{*T}
+ \I)/\phi)  \pause \\
\phi \mid \Y & \sim & \G\left(\frac{\nu_n}{2},
  \frac{\shat \nu_n}{ 2} \right)  \pause \\
\Y^* \mid \Y & \sim & t_{\nu_n}( \X^*\bv_n, \shat (\X^* \Phi_n^{-1} \X))
\end{eqnarray*}

Under limiting prior same as frequentist predictive interval
\end{frame}
\begin{frame}
  \frametitle{Conjugate Priors}
  \begin{definition}
    A class of prior distributions $\cP$ for $\t$ is conjugate for a
    sampling model $p(y \mid \t)$ if for every $p(\t) \in \cP$, $p(\t
    \mid \Y) \in \cP$.
  \end{definition}
\pause
  Advantages: \pause
  \begin{itemize}
  \item Closed form distributions for most quantities; bypass MCMC for
    calculations \pause
  \item Simple updating in terms of sufficient statistics ``weighted
    average'' \pause
  \item Interpretation as prior samples - prior sample size \pause
  \item Elicitation of prior through imaginary or historical data \pause
  \item limiting ``non-proper'' form recovers MLEs \pause
  \end{itemize}

\end{frame}
\begin{frame}
  \frametitle{Some Default Choices}
  \begin{itemize}
  \item Independent Jeffreys Prior $p(\b, \phi) \propto 1/\phi$
  \item Unit information prior $\b \mid \phi \sim \N(\bhat, n
    (\X^T\X)^{-1}/\phi)$ \pause
\item  Zellner's g-prior(s) $\b \mid \phi \sim \N(\bv_0, g
    (\X^T\X)^{-1}/\phi)$ \pause

$$\b \mid \Y, \phi \sim \N\left( \frac{g}{1 + g} \bhat +  \frac{1}{1 + g}
\bv_0, \frac{g}{1 + g} (\X^T\X)^{-1} \phi^{-1} \right)$$ \pause
\item Independent Priors on $\beta_j$
$$\beta_j \mid \phi, \lambda_j \ind N(0, s^2_j /(\phi \lambda_j))$$
\end{itemize}

\end{frame}
\begin{frame}
  \frametitle{Disadvantages}
  Disadvantages: \pause
\begin{itemize}
\item Results  may have be sensitive to prior ``outliers'' due to
  linear updating \pause
\vspace{2.25in}

\item Cannot capture all possible prior beliefs \pause
\end{itemize}


\end{frame}
\begin{frame}
  \frametitle{Mixtures of Conjugate Priors}
  \begin{theorem}[Diaconis \& Ylivisaker 1985]  Given a sampling model
  $p(y \mid \t)$ from an exponential family, any prior distribution
  can be expressed as a mixture of conjugate prior distributions
 \end{theorem}

 \begin{itemize}
 \item Prior $p(\t) = \int p(\t \mid \omega) p(\omega)\, d \omega$ \pause
 \item Posterior \pause
   \begin{eqnarray*}
   p(\t \mid \Y)  &\propto & \int p(\Y \mid \t) p(\t \mid \omega)
   p(\omega) \, d\omega \pause \\
 & \propto & \int  \frac{  p(\Y \mid \t) p(\t \mid \omega) } {p(\Y \mid
   \omega)}  p(\Y \mid
 \omega) p(\omega ) \, d \omega  \pause \\
& \propto & \int p(\t \mid \Y, \omega)  p(\Y \mid
 \omega) p(\omega ) \, d \omega \pause \\
 p(\t \mid \Y) & =  & \frac{\int p(\t \mid \Y, \omega)  p(\Y \mid
 \omega) p(\omega ) \, d \omega }
{\int p(\Y \mid
 \omega) p(\omega ) \, d \omega }
       \end{eqnarray*}

 \end{itemize}
\end{frame}
\begin{frame}
  \frametitle{Next Class After Midterm}
  Review Hoff Chapters 1-9,  Gelman-Hill Chapter 18
\end{frame}
\end{document}

