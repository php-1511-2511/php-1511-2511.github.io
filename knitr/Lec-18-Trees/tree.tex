%\documentclass[]{beamer}
\documentclass[handout]{beamer}\usepackage[]{graphicx}\usepackage[]{color}
%% maxwidth is the original width if it is less than linewidth
%% otherwise use linewidth (to make sure the graphics do not exceed the margin)
\makeatletter
\def\maxwidth{ %
  \ifdim\Gin@nat@width>\linewidth
    \linewidth
  \else
    \Gin@nat@width
  \fi
}
\makeatother

\definecolor{fgcolor}{rgb}{0.345, 0.345, 0.345}
\newcommand{\hlnum}[1]{\textcolor[rgb]{0.686,0.059,0.569}{#1}}%
\newcommand{\hlstr}[1]{\textcolor[rgb]{0.192,0.494,0.8}{#1}}%
\newcommand{\hlcom}[1]{\textcolor[rgb]{0.678,0.584,0.686}{\textit{#1}}}%
\newcommand{\hlopt}[1]{\textcolor[rgb]{0,0,0}{#1}}%
\newcommand{\hlstd}[1]{\textcolor[rgb]{0.345,0.345,0.345}{#1}}%
\newcommand{\hlkwa}[1]{\textcolor[rgb]{0.161,0.373,0.58}{\textbf{#1}}}%
\newcommand{\hlkwb}[1]{\textcolor[rgb]{0.69,0.353,0.396}{#1}}%
\newcommand{\hlkwc}[1]{\textcolor[rgb]{0.333,0.667,0.333}{#1}}%
\newcommand{\hlkwd}[1]{\textcolor[rgb]{0.737,0.353,0.396}{\textbf{#1}}}%
\let\hlipl\hlkwb

\usepackage{framed}
\makeatletter
\newenvironment{kframe}{%
 \def\at@end@of@kframe{}%
 \ifinner\ifhmode%
  \def\at@end@of@kframe{\end{minipage}}%
  \begin{minipage}{\columnwidth}%
 \fi\fi%
 \def\FrameCommand##1{\hskip\@totalleftmargin \hskip-\fboxsep
 \colorbox{shadecolor}{##1}\hskip-\fboxsep
     % There is no \\@totalrightmargin, so:
     \hskip-\linewidth \hskip-\@totalleftmargin \hskip\columnwidth}%
 \MakeFramed {\advance\hsize-\width
   \@totalleftmargin\z@ \linewidth\hsize
   \@setminipage}}%
 {\par\unskip\endMakeFramed%
 \at@end@of@kframe}
\makeatother

\definecolor{shadecolor}{rgb}{.97, .97, .97}
\definecolor{messagecolor}{rgb}{0, 0, 0}
\definecolor{warningcolor}{rgb}{1, 0, 1}
\definecolor{errorcolor}{rgb}{1, 0, 0}
\newenvironment{knitrout}{}{} % an empty environment to be redefined in TeX

\usepackage{alltt}
%\usepackage[dvips]{color}
%\usepackage{beamerprosper}
\usepackage{graphicx}
%\usepackage{psfrag, pstricks}
\usepackage{amsmath,amssymb,array,comment,eucal}
\usepackage{amsmath,amssymb,array,eucal}
\usepackage{xcolor}
\definecolor{beamer@blendedblue}{RGB}{86,155,189}
\definecolor{myblue}{RGB}{12,76,138}
\setbeamercolor{structure}{fg=myblue}
\definecolor{Ftitle}{RGB}{12,76,138}
\definecolor{Descitem}{RGB}{238,238,244}
\definecolor{StdTitle}{RGB}{12,76,138}
\definecolor{StdBody}{RGB}{213,24,0}
\definecolor{StdBody}{RGB}{213,24,0}

\definecolor{AlTitle}{RGB}{255, 190, 190}
\definecolor{AlBody}{RGB}{213,24,0}

\definecolor{ExTitle}{RGB}{201, 217, 217}
\definecolor{ExBody}{RGB}{213,24,0}

\setbeamercolor{frametitle}{fg = Ftitle}
\setbeamercolor{title}{fg = Ftitle}
\setbeamercolor{item}{fg = Ftitle}
\setbeamercolor{subitem}{fg = Ftitle}
\setbeamercolor{subsubitem}{fg = Ftitle}
\setbeamercolor{description item}{fg = myblue}
\setbeamercolor{titlelike}{fg=myblue}
\newcommand{\e}{\mathbf{e}}
\renewcommand{\P}{\textsf{P}}
\newcommand{\R}{\textsf{R}}
\newcommand{\mat}[1] {\mathbf{#1}}
%\newcommand{\ind}{\mathrel{\mathop{\sim}\limits^{\mathit{ind}}}}
%\newcommand{\iid}{\mathrel{\mathop{\sim}\limits^{\mathit{iid}}}}
\newcommand{\E}{\textsf{E}}
\newcommand{\SE}{\textsf{SE}}
\newcommand{\SSE}{\textsf{SSE}}
\renewcommand{\SS}{\textsf{SS}}
\newcommand{\MSE}{\textsf{MSE}}
\newcommand{\SSR}{\textsf{SSR}}
\newcommand{\Be}{\textsf{Beta}}
\newcommand{\St}{\textsf{St}}
%\newcommand{\C}{\textsf{C}}
\newcommand{\GDP}{\textsf{GDP}}
\newcommand{\NcSt}{\textsf{NcSt}}
\newcommand{\Bin}{\textsf{Bin}}
\newcommand{\NB}{\textsf{NegBin}}
\renewcommand{\NG}{\textsf{NG}}
\newcommand{\N}{\textsf{N}}
\newcommand{\Ber}{\textsf{Ber}}
\newcommand{\Poi}{\text{Poi}}
\newcommand{\Gam}{\textsf{Gamma}}
\newcommand{\Gm}{\textsf{G}}
\newcommand{\Un}{\textsf{Unif}}
\newcommand{\Ex}{\textsf{Exp}}
\newcommand{\DE}{\textsf{DE}}
\newcommand{\tr}{\textsf{tr}}
\newcommand{\cF}{{\cal{F}}}
\newcommand{\cL}{{\cal{L}}}
\newcommand{\cI}{{\cal{I}}}
\newcommand{\cB}{{\cal{B}}}
\newcommand{\cP}{{\cal{P}}}
\newcommand{\bbR}{\mathbb{R}}
\newcommand{\bbN}{\mathbb{N}}
\newcommand{\pperp}{\mathrel{{\rlap{$\,\perp$}\perp\,\,}}}
\newcommand{\OFP}{(\Omega,\cF, \P)}
\newcommand{\eps}{\boldsymbol{\epsilon}}
\newcommand{\1}{\mathbf{1}_n}
\newcommand{\gap}{\vspace{8mm}}
\newcommand{\ind}{\mathrel{\mathop{\sim}\limits^{\rm ind}}}
\newcommand{\simiid}{\ensuremath{\mathrel{\mathop{\sim}\limits^{\rm
iid}}}}
\newcommand{\eqindis}{\ensuremath{\mathrel{\mathop{=}\limits^{\rm D}}}}
\newcommand{\iid}{\textit{i.i.d.}}
\newcommand{\SSZ}{S_{zz}}
\newcommand{\SZW}{S_{zw}}
\newcommand{\Bias}{\textsf{Bias}}
\newcommand{\Var}{\textsf{Var}}
\newcommand{\corr}{\textsf{corr}}
\newcommand{\diag}{\textsf{diag}}
\newcommand{\var}{\textsf{var}}
\newcommand{\Cov}{\textsf{Cov}}
\newcommand{\Sam}{{\cal S}}
\def\H{\mathbf{H}}
\newcommand{\I}{\mathbf{I}}
\newcommand{\Y}{\mathbf{Y}}
\newcommand{\tY}{\tilde{\mathbf{Y}}}
\newcommand{\Yhat}{\hat{\mathbf{Y}}}
\newcommand{\Yobs}{\mathbf{Y}_{{\cal S}}}
\newcommand{\barYobs}{\bar{Y}_{{\cal S}}}
\newcommand{\barYmiss}{\bar{Y}_{{\cal S}^c}}
\def\bv{\mathbf{b}}
\def\X{\mathbf{X}}
\def\tX{\tilde{\mathbf{X}}}
\def\x{\mathbf{x}}
\def\xbar{\bar{\x}}
\def\Xg{\mathbf{X}_{\boldsymbol{\gamma}}}
\def\Ybar{\bar{Y}}
\def\ybar{\bar{y}}
\def\y{\mathbf{y}}
\def\Yf{\mathbf{Y_f}}
\def\W{\mathbf{W}}
\def\w{\mathbf{w}}
\def\U{\mathbf{U}}
\def\V{\mathbf{V}}
\def\Q{\mathbf{Q}}
\def\Z{\mathbf{Z}}
\def\z{\mathbf{z}}
\def\v{\mathbf{v}}
\def\u{\mathbf{u}}

\def\zero{\mathbf{0}}
\def\one{\mathbf{1}}
\newcommand{\taub}{\boldsymbol{\tau}}
\newcommand{\betav}{\boldsymbol{\beta}}
\newcommand{\alphav}{\boldsymbol{\alpha}}
\newcommand{\A}{\mathbf{A}}
\def\a{\mathbf{a}}
\newcommand{\B}{\mathbf{B}}
\def\b{\boldsymbol{\beta}}
\def\bhat{\hat{\boldsymbol{\beta}}}
\def\tb{\tilde{\boldsymbol{\beta}}}
\def\bg{\boldsymbol{\beta_\gamma}}
\def\bgnot{\boldsymbol{\beta_{(-\gamma)}}}
\def\mub{\boldsymbol{\mu}}
\def\tmub{\tilde{\boldsymbol{\mu}}}
\def\muhat{\hat{\boldsymbol{\mu}}}
\def\t{\boldsymbol{\theta}}
\def\tk{\boldsymbol{\theta}_k}
\def\tj{\boldsymbol{\theta}_j}
\def\Mk{\boldsymbol{{\cal M}}_k}
\def\M{{{\cal M}}}
\def\Mj{{{\cal M}}_j}
\def\Mi{{{\cal M}}_i}
\def\Mg{{\cal M}_\gamma}
\def\Mnull{{\cal M}_{N}}
\def\gMPM{\boldsymbol{\gamma}_{\text{MPM}}}
\def\gHPM{\boldsymbol{\gamma}_{\text{HPM}}}
\def\Mfull{\boldsymbol{{\cal M}}_{F}}
\def\tg{\boldsymbol{\theta}_{\boldsymbol{\gamma}}}
\def\g{\boldsymbol{\gamma}}
\def\eg{\boldsymbol{\eta}_{\boldsymbol{\gamma}}}
\def\G{\mathbf{G}}
\def\cM{\cal M}
\def\D{\Delta}
\def \shat{{\hat{\sigma}}^2}
\def\uv{\mathbf{u}}
\def\l {\lambda}
\def\d{\delta}
\def\Sigmab{\boldsymbol{\Sigma}}
\def\Lambdab{\boldsymbol{\Lambda}}
\def\lambdab{\boldsymbol{\lambda}}
\def\Mg{{\cal M}_\gamma}
\def\S{{\cal{S}}}
\def\qg{p_{\boldsymbol{\gamma}}}
\def\pg{p_{\boldsymbol{\gamma}}}
\def\t{\boldsymbol{\theta}}
\def\T{\boldsymbol{\Theta}}

\usepackage{verbatim}
% abbreviation
\def\logit{\textsf{logit}}


\title{Introduction to Classification \& Regression Trees}

\author{ISLR Chapter 8 }
\date{\today}
\IfFileExists{upquote.sty}{\usepackage{upquote}}{}
\begin{document}
\maketitle
\begin{frame}[fragile]\frametitle{Classification and Regression Trees}

Carseat data from ISLR package \pause

\begin{itemize}
\item Binary Outcome {\tt High} 1 if Sales $ > 8$, otherwise  0 \pause
\item Fit a Classification tree model to {\tt Price} and {\tt Income} \pause
\item Pick a predictor  and a cutpoint to split data
$$  X_j \le s  \text{ and } X_k > s$$
to minimize deviance (or SSE for regression) - leads to a  root node in a tree  \pause
\item continue  splitting/partitioning data until stopping criterion is reached  (number of observations in a node $ > 10$ and within node deviance $ > 0.01$ deviance of the root node) \pause
\item  Prediction is mean or proportion of
  successes of data in terminal nodes \pause
\item Output is a decision tree \pause
\item regression or classification function is nonlinear in predictors \pause
\item Captures interactions
\end{itemize}

\end{frame}



\begin{frame}[fragile] \frametitle{Carseat Example}


\begin{knitrout}
\definecolor{shadecolor}{rgb}{0.969, 0.969, 0.969}\color{fgcolor}\begin{kframe}
\begin{alltt}
\hlkwd{library}\hlstd{(tree)}
\hlkwd{data}\hlstd{(Carseats)}
  \hlstd{Carseats} \hlkwb{=} \hlkwd{mutate}\hlstd{(Carseats,} \hlkwc{High}\hlstd{=}\hlkwd{factor}\hlstd{(}\hlkwd{ifelse} \hlstd{(}
  \hlstd{Carseats}\hlopt{$}\hlstd{Sales} \hlopt{<=}\hlnum{8}\hlstd{,}
    \hlstr{"No"}\hlstd{,}
    \hlstr{"Yes "}\hlstd{))}
  \hlstd{)}

\hlstd{tree.carseats} \hlkwb{=} \hlkwd{tree}\hlstd{(High} \hlopt{~} \hlstd{Price} \hlopt{+} \hlstd{Income,}
                     \hlkwc{data}\hlstd{=Carseats)}
\end{alltt}
\end{kframe}
\end{knitrout}

\end{frame}

\begin{frame}[fragile] \frametitle{Carseat Example}
\begin{knitrout}
\definecolor{shadecolor}{rgb}{0.969, 0.969, 0.969}\color{fgcolor}\begin{kframe}
\begin{alltt}
 \hlkwd{plot}\hlstd{(tree.carseats)}
 \hlkwd{text}\hlstd{(tree.carseats)}
\end{alltt}
\end{kframe}

{\centering \includegraphics[width=\maxwidth]{figure/tree-1-1} 

}



\end{knitrout}

\end{frame}


\begin{frame}[fragile]\frametitle{Partition}

\begin{knitrout}
\definecolor{shadecolor}{rgb}{0.969, 0.969, 0.969}\color{fgcolor}\begin{kframe}
\begin{alltt}
\hlkwd{partition.tree}\hlstd{(tree.carseats)}
\hlkwd{points}\hlstd{(Carseats}\hlopt{$}\hlstd{Price,Carseats}\hlopt{$}\hlstd{Income,} \hlkwc{col}\hlstd{=Carseats}\hlopt{$}\hlstd{High)}
\end{alltt}
\end{kframe}

{\centering \includegraphics[width=\maxwidth]{figure/partition-1} 

}



\end{knitrout}

\end{frame}

\begin{frame}[fragile] \frametitle{Splits}
\begin{knitrout}
\definecolor{shadecolor}{rgb}{0.969, 0.969, 0.969}\color{fgcolor}\begin{kframe}
\begin{alltt}
\hlstd{tree.carseats}
\end{alltt}
\begin{verbatim}
## node), split, n, deviance, yval, (yprob)
##       * denotes terminal node
## 
##  1) root 400 541.50 No ( 0.5900 0.4100 )  
##    2) Price < 92.5 62  66.24 Yes  ( 0.2258 0.7742 ) *
##    3) Price > 92.5 338 434.80 No ( 0.6568 0.3432 )  
##      6) Price < 142 287 382.10 No ( 0.6167 0.3833 )  
##       12) Income < 60.5 113 128.70 No ( 0.7434 0.2566 ) *
##       13) Income > 60.5 174 240.40 No ( 0.5345 0.4655 ) *
##      7) Price > 142 51  36.95 No ( 0.8824 0.1176 )  
##       14) Income < 62.5 19   0.00 No ( 1.0000 0.0000 ) *
##       15) Income > 62.5 32  30.88 No ( 0.8125 0.1875 ) *
\end{verbatim}
\end{kframe}
\end{knitrout}
\end{frame}




\begin{frame}[fragile]\frametitle{Summary}
\begin{knitrout}
\definecolor{shadecolor}{rgb}{0.969, 0.969, 0.969}\color{fgcolor}\begin{kframe}
\begin{alltt}
 \hlkwd{summary}\hlstd{(tree.carseats)}
\end{alltt}
\begin{verbatim}
## 
## Classification tree:
## tree(formula = High ~ Price + Income, data = Carseats)
## Number of terminal nodes:  5 
## Residual mean deviance:  1.18 = 466.2 / 395 
## Misclassification error rate: 0.325 = 130 / 400
\end{verbatim}
\end{kframe}
\end{knitrout}
\end{frame}


\begin{frame}[fragile]{All Variables}

\begin{knitrout}
\definecolor{shadecolor}{rgb}{0.969, 0.969, 0.969}\color{fgcolor}\begin{kframe}
\begin{alltt}
\hlstd{tree.carseats} \hlkwb{=}\hlkwd{tree}\hlstd{(High} \hlopt{~}  \hlstd{.} \hlopt{-}\hlstd{Sales,} \hlkwc{data}\hlstd{=Carseats )}
\hlkwd{summary}\hlstd{(tree.carseats)}
\end{alltt}
\begin{verbatim}
## 
## Classification tree:
## tree(formula = High ~ . - Sales, data = Carseats)
## Variables actually used in tree construction:
## [1] "ShelveLoc"   "Price"       "Income"      "CompPrice"   "Population" 
## [6] "Advertising" "Age"         "US"         
## Number of terminal nodes:  27 
## Residual mean deviance:  0.4575 = 170.7 / 373 
## Misclassification error rate: 0.09 = 36 / 400
\end{verbatim}
\end{kframe}
\end{knitrout}

Overfitting?
\end{frame}

\begin{frame}\frametitle{Tree}
\vspace{-.5in}
\begin{knitrout}
\definecolor{shadecolor}{rgb}{0.969, 0.969, 0.969}\color{fgcolor}

{\centering \includegraphics[width=\maxwidth]{figure/tree-plot-1} 

}



\end{knitrout}

\end{frame}

\begin{frame}[fragile]\frametitle{Classification Error}
\begin{knitrout}
\definecolor{shadecolor}{rgb}{0.969, 0.969, 0.969}\color{fgcolor}\begin{kframe}
\begin{alltt}
\hlkwd{set.seed} \hlstd{(}\hlnum{2}\hlstd{)}
\hlstd{train}\hlkwb{=}\hlkwd{sample} \hlstd{(}\hlnum{1}\hlopt{:} \hlkwd{nrow}\hlstd{(Carseats ),} \hlnum{200}\hlstd{)}
\hlstd{Carseats.test}\hlkwb{=}\hlstd{Carseats [}\hlopt{-}\hlstd{train ,]}

\hlstd{tree.carseats} \hlkwb{=}\hlkwd{tree}\hlstd{(High} \hlopt{~}  \hlstd{.} \hlopt{-}\hlstd{Sales,}
                    \hlkwc{data}\hlstd{=Carseats,}\hlkwc{subset}\hlstd{=train )}
\hlstd{tree.pred}\hlkwb{=}\hlkwd{predict} \hlstd{(tree.carseats ,Carseats.test ,}\hlkwc{type} \hlstd{=}\hlstr{"class"}\hlstd{)}
\hlkwd{table}\hlstd{(tree.pred , Carseats.test}\hlopt{$}\hlstd{High)}
\end{alltt}
\begin{verbatim}
##          
## tree.pred No Yes 
##      No   86   27
##      Yes  30   57
\end{verbatim}
\end{kframe}
\end{knitrout}


\begin{knitrout}
\definecolor{shadecolor}{rgb}{0.969, 0.969, 0.969}\color{fgcolor}\begin{kframe}
\begin{alltt}
 \hlstd{(}\hlnum{30} \hlopt{+} \hlnum{27}\hlstd{)} \hlopt{/}\hlnum{200}   \hlcom{# classification error}
\end{alltt}
\begin{verbatim}
## [1] 0.285
\end{verbatim}
\end{kframe}
\end{knitrout}

\end{frame}
\begin{frame}\frametitle{Cost-Complexity Pruning}
\begin{enumerate}
\item Grow a large tree on training data, stopping when each terminal node has fewer than some minimum number of observations \pause
\item Prediction for region $m$ is the Class $c$ that $max_{c} \hat{\pi}_{m c}$ \pause
\item  Snip  off the least important splits via  cost-complexity pruning to the tree in order to obtain a sequence of best subtrees indexed by cost parameter $k$,  \pause

$$\frac{N_{miss}}{N} - k |T| $$
 missclassification error penalized  by number of terminal nodes  \pause
\item Using $K$-fold cross validation, compute average cost-complexity for each $k$  \pause
\item Pick subtree with smallest penalized error
\end{enumerate}
\end{frame}

\begin{frame}[fragile]{Pruning via Cross Validation )}
\begin{knitrout}
\definecolor{shadecolor}{rgb}{0.969, 0.969, 0.969}\color{fgcolor}\begin{kframe}
\begin{alltt}
\hlkwd{set.seed}\hlstd{(}\hlnum{2}\hlstd{)}
\hlstd{cv.carseats} \hlkwb{=} \hlkwd{cv.tree}\hlstd{(tree.carseats,} \hlkwc{FUN}\hlstd{=prune.misclass)}
\end{alltt}
\end{kframe}
\end{knitrout}

\vspace{-.5in}
\begin{knitrout}
\definecolor{shadecolor}{rgb}{0.969, 0.969, 0.969}\color{fgcolor}
\includegraphics[width=\maxwidth]{figure/cv_plot-1} 

\end{knitrout}

\end{frame}




\begin{frame}[fragile]\frametitle{Pruned}

\begin{knitrout}
\definecolor{shadecolor}{rgb}{0.969, 0.969, 0.969}\color{fgcolor}\begin{kframe}
\begin{alltt}
\hlstd{prune.carseats} \hlkwb{=} \hlkwd{prune.misclass}\hlstd{(tree.carseats ,}\hlkwc{best} \hlstd{=}\hlnum{9}\hlstd{)}
\end{alltt}
\end{kframe}
\end{knitrout}

\begin{knitrout}
\definecolor{shadecolor}{rgb}{0.969, 0.969, 0.969}\color{fgcolor}
\includegraphics[width=\maxwidth]{figure/plot_pruned-1} 

\end{knitrout}

\end{frame}


\begin{frame}[fragile]\frametitle{Miss-classification after Selection}

\begin{knitrout}
\definecolor{shadecolor}{rgb}{0.969, 0.969, 0.969}\color{fgcolor}\begin{kframe}
\begin{alltt}
\hlstd{tree.pred}\hlkwb{=}\hlkwd{predict}\hlstd{(prune.carseats , Carseats.test,}
                  \hlkwc{type}\hlstd{=}\hlstr{"class"}\hlstd{)}
\hlkwd{table}\hlstd{(tree.pred ,Carseats.test}\hlopt{$}\hlstd{High)}
\end{alltt}
\begin{verbatim}
##          
## tree.pred No Yes 
##      No   94   24
##      Yes  22   60
\end{verbatim}
\end{kframe}
\end{knitrout}

\begin{knitrout}
\definecolor{shadecolor}{rgb}{0.969, 0.969, 0.969}\color{fgcolor}\begin{kframe}
\begin{alltt}
\hlstd{(}\hlnum{94} \hlopt{+}\hlnum{60}\hlstd{)}\hlopt{/}\hlnum{200}  \hlcom{# classified Correctly}
\end{alltt}
\begin{verbatim}
## [1] 0.77
\end{verbatim}
\end{kframe}
\end{knitrout}

\end{frame}



\begin{frame}\frametitle{Next Class}
Trees are simple to understand, but not as competitive with other supervised learning approaches for prediction/classification. \pause

Ways to improving Trees through multiple trees in some ensemble:
  \begin{itemize}
  \item Bagging   \pause
  \item Random Forests  \pause
  \item Boosting  \pause
  \item BART  (Bayesian Additive Regression Trees)  \pause
  \end{itemize}
Combining trees will yield improved prediction accuracy, but with loss of interpretability.


\end{frame}
\end{document}
