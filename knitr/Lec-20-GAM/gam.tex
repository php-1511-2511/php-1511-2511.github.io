\documentclass[]{beamer}\usepackage[]{graphicx}\usepackage[]{color}
%% maxwidth is the original width if it is less than linewidth
%% otherwise use linewidth (to make sure the graphics do not exceed the margin)
\makeatletter
\def\maxwidth{ %
  \ifdim\Gin@nat@width>\linewidth
    \linewidth
  \else
    \Gin@nat@width
  \fi
}
\makeatother

\definecolor{fgcolor}{rgb}{0.345, 0.345, 0.345}
\newcommand{\hlnum}[1]{\textcolor[rgb]{0.686,0.059,0.569}{#1}}%
\newcommand{\hlstr}[1]{\textcolor[rgb]{0.192,0.494,0.8}{#1}}%
\newcommand{\hlcom}[1]{\textcolor[rgb]{0.678,0.584,0.686}{\textit{#1}}}%
\newcommand{\hlopt}[1]{\textcolor[rgb]{0,0,0}{#1}}%
\newcommand{\hlstd}[1]{\textcolor[rgb]{0.345,0.345,0.345}{#1}}%
\newcommand{\hlkwa}[1]{\textcolor[rgb]{0.161,0.373,0.58}{\textbf{#1}}}%
\newcommand{\hlkwb}[1]{\textcolor[rgb]{0.69,0.353,0.396}{#1}}%
\newcommand{\hlkwc}[1]{\textcolor[rgb]{0.333,0.667,0.333}{#1}}%
\newcommand{\hlkwd}[1]{\textcolor[rgb]{0.737,0.353,0.396}{\textbf{#1}}}%
\let\hlipl\hlkwb

\usepackage{framed}
\makeatletter
\newenvironment{kframe}{%
 \def\at@end@of@kframe{}%
 \ifinner\ifhmode%
  \def\at@end@of@kframe{\end{minipage}}%
  \begin{minipage}{\columnwidth}%
 \fi\fi%
 \def\FrameCommand##1{\hskip\@totalleftmargin \hskip-\fboxsep
 \colorbox{shadecolor}{##1}\hskip-\fboxsep
     % There is no \\@totalrightmargin, so:
     \hskip-\linewidth \hskip-\@totalleftmargin \hskip\columnwidth}%
 \MakeFramed {\advance\hsize-\width
   \@totalleftmargin\z@ \linewidth\hsize
   \@setminipage}}%
 {\par\unskip\endMakeFramed%
 \at@end@of@kframe}
\makeatother

\definecolor{shadecolor}{rgb}{.97, .97, .97}
\definecolor{messagecolor}{rgb}{0, 0, 0}
\definecolor{warningcolor}{rgb}{1, 0, 1}
\definecolor{errorcolor}{rgb}{1, 0, 0}
\newenvironment{knitrout}{}{} % an empty environment to be redefined in TeX

\usepackage{alltt}
%\documentclass[handout]{beamer}
%\usepackage[dvips]{color}
%\usepackage{beamerprosper}
\usepackage{graphicx}
%\usepackage{psfrag, pstricks}
\usepackage{amsmath,amssymb,array,comment,eucal}
\usepackage{amsmath,amssymb,array,eucal}
\usepackage{xcolor}
\definecolor{beamer@blendedblue}{RGB}{86,155,189}
\definecolor{myblue}{RGB}{12,76,138}
\setbeamercolor{structure}{fg=myblue}
\definecolor{Ftitle}{RGB}{12,76,138}
\definecolor{Descitem}{RGB}{238,238,244}
\definecolor{StdTitle}{RGB}{12,76,138}
\definecolor{StdBody}{RGB}{213,24,0}
\definecolor{StdBody}{RGB}{213,24,0}

\definecolor{AlTitle}{RGB}{255, 190, 190}
\definecolor{AlBody}{RGB}{213,24,0}

\definecolor{ExTitle}{RGB}{201, 217, 217}
\definecolor{ExBody}{RGB}{213,24,0}

\setbeamercolor{frametitle}{fg = Ftitle}
\setbeamercolor{title}{fg = Ftitle}
\setbeamercolor{item}{fg = Ftitle}
\setbeamercolor{subitem}{fg = Ftitle}
\setbeamercolor{subsubitem}{fg = Ftitle}
\setbeamercolor{description item}{fg = myblue}
\setbeamercolor{titlelike}{fg=myblue}
\newcommand{\e}{\mathbf{e}}
\renewcommand{\P}{\textsf{P}}
\newcommand{\R}{\textsf{R}}
\newcommand{\mat}[1] {\mathbf{#1}}
%\newcommand{\ind}{\mathrel{\mathop{\sim}\limits^{\mathit{ind}}}}
%\newcommand{\iid}{\mathrel{\mathop{\sim}\limits^{\mathit{iid}}}}
\newcommand{\E}{\textsf{E}}
\newcommand{\SE}{\textsf{SE}}
\newcommand{\SSE}{\textsf{SSE}}
\renewcommand{\SS}{\textsf{SS}}
\newcommand{\MSE}{\textsf{MSE}}
\newcommand{\SSR}{\textsf{SSR}}
\newcommand{\Be}{\textsf{Beta}}
\newcommand{\St}{\textsf{St}}
%\newcommand{\C}{\textsf{C}}
\newcommand{\GDP}{\textsf{GDP}}
\newcommand{\NcSt}{\textsf{NcSt}}
\newcommand{\Bin}{\textsf{Bin}}
\newcommand{\NB}{\textsf{NegBin}}
\renewcommand{\NG}{\textsf{NG}}
\newcommand{\N}{\textsf{N}}
\newcommand{\Ber}{\textsf{Ber}}
\newcommand{\Poi}{\text{Poi}}
\newcommand{\Gam}{\textsf{Gamma}}
\newcommand{\Gm}{\textsf{G}}
\newcommand{\Un}{\textsf{Unif}}
\newcommand{\Ex}{\textsf{Exp}}
\newcommand{\DE}{\textsf{DE}}
\newcommand{\tr}{\textsf{tr}}
\newcommand{\cF}{{\cal{F}}}
\newcommand{\cL}{{\cal{L}}}
\newcommand{\cI}{{\cal{I}}}
\newcommand{\cB}{{\cal{B}}}
\newcommand{\cP}{{\cal{P}}}
\newcommand{\bbR}{\mathbb{R}}
\newcommand{\bbN}{\mathbb{N}}
\newcommand{\pperp}{\mathrel{{\rlap{$\,\perp$}\perp\,\,}}}
\newcommand{\OFP}{(\Omega,\cF, \P)}
\newcommand{\eps}{\boldsymbol{\epsilon}}
\newcommand{\1}{\mathbf{1}_n}
\newcommand{\gap}{\vspace{8mm}}
\newcommand{\ind}{\mathrel{\mathop{\sim}\limits^{\rm ind}}}
\newcommand{\simiid}{\ensuremath{\mathrel{\mathop{\sim}\limits^{\rm
iid}}}}
\newcommand{\eqindis}{\ensuremath{\mathrel{\mathop{=}\limits^{\rm D}}}}
\newcommand{\iid}{\textit{i.i.d.}}
\newcommand{\SSZ}{S_{zz}}
\newcommand{\SZW}{S_{zw}}
\newcommand{\Bias}{\textsf{Bias}}
\newcommand{\Var}{\textsf{Var}}
\newcommand{\corr}{\textsf{corr}}
\newcommand{\diag}{\textsf{diag}}
\newcommand{\var}{\textsf{var}}
\newcommand{\Cov}{\textsf{Cov}}
\newcommand{\Sam}{{\cal S}}
\def\H{\mathbf{H}}
\newcommand{\I}{\mathbf{I}}
\newcommand{\Y}{\mathbf{Y}}
\newcommand{\tY}{\tilde{\mathbf{Y}}}
\newcommand{\Yhat}{\hat{\mathbf{Y}}}
\newcommand{\Yobs}{\mathbf{Y}_{{\cal S}}}
\newcommand{\barYobs}{\bar{Y}_{{\cal S}}}
\newcommand{\barYmiss}{\bar{Y}_{{\cal S}^c}}
\def\bv{\mathbf{b}}
\def\X{\mathbf{X}}
\def\tX{\tilde{\mathbf{X}}}
\def\x{\mathbf{x}}
\def\xbar{\bar{\x}}
\def\Xg{\mathbf{X}_{\boldsymbol{\gamma}}}
\def\Ybar{\bar{Y}}
\def\ybar{\bar{y}}
\def\y{\mathbf{y}}
\def\Yf{\mathbf{Y_f}}
\def\W{\mathbf{W}}
\def\w{\mathbf{w}}
\def\U{\mathbf{U}}
\def\V{\mathbf{V}}
\def\Q{\mathbf{Q}}
\def\Z{\mathbf{Z}}
\def\z{\mathbf{z}}
\def\v{\mathbf{v}}
\def\u{\mathbf{u}}

\def\zero{\mathbf{0}}
\def\one{\mathbf{1}}
\newcommand{\taub}{\boldsymbol{\tau}}
\newcommand{\betav}{\boldsymbol{\beta}}
\newcommand{\alphav}{\boldsymbol{\alpha}}
\newcommand{\A}{\mathbf{A}}
\def\a{\mathbf{a}}
\newcommand{\B}{\mathbf{B}}
\def\b{\boldsymbol{\beta}}
\def\bhat{\hat{\boldsymbol{\beta}}}
\def\tb{\tilde{\boldsymbol{\beta}}}
\def\bg{\boldsymbol{\beta_\gamma}}
\def\bgnot{\boldsymbol{\beta_{(-\gamma)}}}
\def\mub{\boldsymbol{\mu}}
\def\tmub{\tilde{\boldsymbol{\mu}}}
\def\muhat{\hat{\boldsymbol{\mu}}}
\def\t{\boldsymbol{\theta}}
\def\tk{\boldsymbol{\theta}_k}
\def\tj{\boldsymbol{\theta}_j}
\def\Mk{\boldsymbol{{\cal M}}_k}
\def\M{{{\cal M}}}
\def\Mj{{{\cal M}}_j}
\def\Mi{{{\cal M}}_i}
\def\Mg{{\cal M}_\gamma}
\def\Mnull{{\cal M}_{N}}
\def\gMPM{\boldsymbol{\gamma}_{\text{MPM}}}
\def\gHPM{\boldsymbol{\gamma}_{\text{HPM}}}
\def\Mfull{\boldsymbol{{\cal M}}_{F}}
\def\tg{\boldsymbol{\theta}_{\boldsymbol{\gamma}}}
\def\g{\boldsymbol{\gamma}}
\def\eg{\boldsymbol{\eta}_{\boldsymbol{\gamma}}}
\def\G{\mathbf{G}}
\def\cM{\cal M}
\def\D{\Delta}
\def \shat{{\hat{\sigma}}^2}
\def\uv{\mathbf{u}}
\def\l {\lambda}
\def\d{\delta}
\def\Sigmab{\boldsymbol{\Sigma}}
\def\Lambdab{\boldsymbol{\Lambda}}
\def\lambdab{\boldsymbol{\lambda}}
\def\Mg{{\cal M}_\gamma}
\def\S{{\cal{S}}}
\def\qg{p_{\boldsymbol{\gamma}}}
\def\pg{p_{\boldsymbol{\gamma}}}
\def\t{\boldsymbol{\theta}}
\def\T{\boldsymbol{\Theta}}

\usepackage{verbatim}
% abbreviation
\def\logit{\textsf{logit}}




\title{Introduction to Generalized Additive Models}

\author{ISLR Chapter 7 }
\date{\today}
\IfFileExists{upquote.sty}{\usepackage{upquote}}{}
\begin{document}
\maketitle



\begin{frame}[fragile]\frametitle{Moving Beyond Linearity}
Wage data from ISLR   {\tt data(Wage)}

Residual plots from Simple Linear Regression of Wage on Age:


\centerline{\includegraphics[height=3in]{resid-slr}}
\end{frame}

\begin{frame}[fragile]\frametitle{Polynomial Regression}
\begin{verbatim}
> summary(lm(wage ~ age + I(age^2) + I(age^3) + I(age^4),
             data=Wage))
Coefficients:
              Estimate Std. Error t value Pr(>|t|)
(Intercept) -1.842e+02  6.004e+01  -3.067 0.002180 **
age          2.125e+01  5.887e+00   3.609 0.000312 ***
I(age^2)    -5.639e-01  2.061e-01  -2.736 0.006261 **
I(age^3)     6.811e-03  3.066e-03   2.221 0.026398 *
I(age^4)    -3.204e-05  1.641e-05  -1.952 0.051039 .
---
Signif. codes:  0 ‘***’ 0.001 ‘**’ 0.01 ‘*’ 0.05 ‘.’ 0.1 ‘ ’ 1

Residual standard error: 39.91 on 2995 degrees of freedom
Multiple R-squared:  0.08626,	Adjusted R-squared:  0.08504
F-statistic: 70.69 on 4 and 2995 DF,  p-value: < 2.2e-16
\end{verbatim}

\end{frame}
\begin{frame}[fragile]{Orthogonal Polynomial}
\begin{verbatim}
> summary(lm(wage ~ poly(age,4),data=Wage))

Coefficients:
               Estimate Std. Error t value Pr(>|t|)
(Intercept)    111.7036     0.7287 153.283  < 2e-16 ***
poly(age, 4)1  447.0679    39.9148  11.201  < 2e-16 ***
poly(age, 4)2 -478.3158    39.9148 -11.983  < 2e-16 ***
poly(age, 4)3  125.5217    39.9148   3.145  0.00168 **
poly(age, 4)4  -77.9112    39.9148  -1.952  0.05104 .
---

Residual standard error: 39.91 on 2995 degrees of freedom
Multiple R-squared:  0.08626,	Adjusted R-squared:  0.08504
F-statistic: 70.69 on 4 and 2995 DF,  p-value: < 2.2e-16
\end{verbatim}
\end{frame}
\begin{frame}\frametitle{Fitted Values}
  \centerline{\includegraphics[height=3in]{poly}}
\end{frame}

\begin{frame}\frametitle{Problems}
  \begin{itemize}
  \item Higher order terms may be needed to fit data globally \pause
  \item generally do not go above 3rd order or 4th order polynomial -
        may be too flexible \pause
  \item fit piece-wise polynomials over different ranges \pause
  \item is function continuous where they join? \pause
  \item is function differentiable where they join? \pause
  \end{itemize}
Add constraints - lose degrees of freedom
\end{frame}

\begin{frame}\frametitle{Piecewise Polynomials}

\centerline{\includegraphics[height=3in]{7-3}}

\end{frame}
\begin{frame}\frametitle{Spline Basis}
Alternative way to represent the model so that we have continuity, continuous first and
second derivatives  is
$$Y_i = \beta_0 + \beta_1 x_i + \beta_2 x_1^2 + \beta_3 x_i^3 + h(x_i,\xi) \beta_4 + \epsilon_i$$

where $\xi$ is a ``knot''' in a truncated cubic basis function
$$
h(x_i, \xi) \equiv (x_i - \xi)^3_+ =  \left\{
\begin{array}{ll}
  (x_i - \xi)^3 & \text{ if }  x_i > \xi \\
  0 & \text{ otherwise}
\end{array} \right.
$$\pause

We can add additional terms that each with 1 degree of freedom
$$Y_i = \beta_0 + \beta_1 x_i + \beta_2 x_1^2 + \beta_3 x_i^3 +
\sum_k^{K}h(x_i,\xi_k) \beta_4 + \epsilon_i$$ \pause


\end{frame}

\begin{frame}\frametitle{Splines}

  \begin{itemize}
  \item B-splines  (reduces multicollinearity between terms from
    truncated basis) {\tt splines} package in R to use {\tt bs(x)} to
    construct basis \pause
  \item natural splines   (add more constraints so that function is linear
    outside range of data) \pause
\item smoothing splines \pause
  \end{itemize}

Choice of knots and/or degrees of freedom? \pause   Smoothing splines place
a knot at each data point, but adds
 a penalty to prevent over-fitting:

$$ \sum (Y_i - g(x_i))^2 +  \lambda \int g^{"}(t)^2 \, dt $$
\pause
This can be reformulated as a Bayesian model with a Gaussian g-prior.
packages use LOOCV or GCV to choose $\lambda$
\end{frame}

\begin{frame}[fragile]\frametitle{Fitting GAMs in R}
\begin{verbatim}
> wage.gam = gam(wage ~ s(age), data=Wage)
> summary(wage.gam)
Parametric coefficients:
            Estimate Std. Error t value Pr(>|t|)
(Intercept) 111.7036     0.7282   153.4   <2e-16 ***

Approximate significance of smooth terms:
         edf Ref.df     F p-value
s(age) 5.298  6.399 44.34  <2e-16 ***
---
Signif. codes:  0 ‘***’ 0.001 ‘**’ 0.01 ‘*’ 0.05 ‘.’ 0.1 ‘ ’ 1

R-sq.(adj) =  0.0864   Deviance explained =  8.8%
GCV = 1594.2  Scale est. = 1590.9    n = 3000
\end{verbatim}
\end{frame}
\begin{frame}\frametitle{Fitted Curve}
{\tt  plot(wage.gam, rug=T) }
\centerline{\includegraphics[height=3in]{gam-age}}
\end{frame}

\begin{frame}[fragile]\frametitle{More terms}
\begin{verbatim}
> wage.gam2 = gam(wage ~ s(year,k=7) + s(age), data=Wage)
> summary(wage.gam2)

Parametric coefficients:
            Estimate Std. Error t value Pr(>|t|)
(Intercept) 111.7036     0.7266   153.7   <2e-16 ***
---

Approximate significance of smooth terms:
          edf Ref.df     F  p-value
s(year) 1.000  1.000 14.18 0.000169 ***
s(age)  5.462  6.568 43.37  < 2e-16 ***

R-sq.(adj) =  0.0905   Deviance explained = 9.24%
GCV = 1587.7  Scale est. = 1583.7    n = 3000
\end{verbatim}
\end{frame}
\begin{frame}\frametitle{Year and Age Smooth fits}
  \centerline{\includegraphics[height=3in]{gam-age-year}}
\end{frame}

\begin{frame}[fragile]\frametitle{Showing Factors}
\begin{verbatim}
wage.gam3 = gam(wage ~ s(year,k=7) + s(age) + education,
                data=Wage)
termplot(wage.gam3, se=T, rug=T, ask=F, col.se=2)
\end{verbatim}
\centerline{\includegraphics[height=2.5in]{edu-term}}
\end{frame}

\begin{frame}\frametitle{Summary}
GAMS:

\begin{itemize}
\item Allow flexible non-linear functions of predictors.  Do not need to try various transformations or polymomials to capture relationships \pause

\item May be used to suggest parametric models (i.e linear or quadratic may be fine) \pause

\item nonlinear functions can extend to multiple predictors for interactions, but soon run into curse of dimensionality \pause

\item Nonlinear fits can lead to improved prediction \pause

\item Additive functions may be more interpretable
\end{itemize}
\end{frame}
\end{document}
